\graphicspath{{content/chapters/1_introduction/figures}}
\chapter{Introduction}
\label{chp:introduction}

Machine learning, commonly referred to as Artificial Intelligence (AI), has seen rapid advancements over the past decade. Largely driven by increasing computational power and the availability of large datasets. These techniques are now widely applied across various domains to solve complex problems, including speech processing and noise cancellation. However, the adoption of AI in such areas often lacks thorough evaluation, with the justifications for their computational demands not always clearly addressed.

This project focuses on noise cancellation in audio signals for speech enhancement. An area with growing importance in telecommunications, assistive technologies, and real-time communication systems. Background noise can significantly degrade the quality of speech, making it difficult to interpret or process spoken information. Such noise is generally categorized into two types: stationary and non-stationary \cite{loizou2013speech}.

\begin{itemize}
    \item \textbf{Stationary noise} has relatively constant properties over time, such as white noise or the hum of an appliance.
    \item \textbf{Non-stationary noise} fluctuates unpredictably, such as traffic, crowd chatter, or sudden environmental sounds.
\end{itemize}

Speech enhancement aims to suppress these unwanted noise components while preserving speech intelligibility. Classical noise cancellation techniques, such as the Wiener filter, work by estimating and subtracting noise from the signal, but they often assume that the noise is stationary. This assumption limits their effectiveness in real-world conditions. 

Machine learning approaches, on the other hand, have emerged as powerful alternatives. Capable of learning complex, data-driven patterns to separate speech from noise without relying on such rigid assumptions.

\section{Project Goals and Implementation}

The primary goal of the project is to explore both classical and machine learning based approaches to noise cancellation, evaluating their effectiveness and feasibility in real-world applications. The project will not only examine established noise reduction techniques but also develop a machine learning based model, comparing its performance against classical methods.

The approach involves designing and implementing a noise cancellation system that operates under the assumption of a single-speaker scenario in a noisy background environment, where the system must remove both stationary and non-stationary noise without access to a clean reference signal. Unlike established pre-trained models that required extensive datasets and resources, the model developed in this project will not be pre-trained. Allowing for a demonstration of the ease of developing and training in practical settings. Pre-trained models will be as part of the evaluation process, but the focus will be on the model developed in this project.

The project will be implemented in Python using a modular and reproducible structure, ensuring that the framework can be extended or modified for further improvements. The project is structured into two core phases training and denoising:

\begin{itemize}
    \item Training phase: The most computationally intensive stage, where a sourced dataset of clean and noisy speech samples will be formatted and fed into the model to learn respective mapping.
    \item Denoising phase: Involves loading the trained model and applying it to a noisy speech signal to generate a cleaned output. Here evalution metrics will also be producable to assess the performance of one methodology against another.
\end{itemize}


\section{Overview of the Contents of the Report}