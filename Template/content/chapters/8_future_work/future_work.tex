\chapter{Future Work}
\label{chap:future_work}

While the project successfully met its primary goal of developing and evaluating a modular speech enhancement system. There remains considerable scope for future work. The pipeline was deliberately designed for modularity, allowing straightforward integration of emerging model architectures. Recent transformer based and diffusion based speech enhancement methods could be incorporated into the existing framework and evaluated using the same metrics. These models may offer gains in both perceptual and numerical performance.

For instance, the transformer based ScaleFormer model \cite{wu2023scaleformer}
is a very powerful architecture used in other domains. It was not implemented due to time constraints and the already high performance achieved, but its results indicate promising future potential. Effectively adapting such models to the spectrogram-based pipeline used here will be key to their success.

Another area for improvement is generalisation. As shown in Appendix~\ref{sec:pretrained_comparison}, both our best model and many pretrained systems struggled with unseen noise types. One outlier, \texttt{DeepFilterNet}, likely benefited from training on a more diverse dataset. This underscores the importance of exposing models to broader training data to ensure better real-world robustness. Since the pipeline was built from scratch. More advanced learning strategies such as reinforcement learning or self-supervised pretraining were not explored. These could accelerate convergence and improve generalisation, especially for larger models or more complex tasks.

Deployment-oriented enhancements also merit investigation. Real devices like headphones and smartphones often use microphone arrays, enabling spatial filtering. Future work could incorporate multichannel input and beamforming to better exploit these cues. Additionally, increased market interest and research into systems integrating playback time feedback or \gls{anc} has been seen. Its development does not reach too far from the current pipeline and evaluation could provide valuable insights.

Overall, the flexibility of the current system lays a strong foundation for continued experimentation and development. By extending its capabilities with newer architectures, more diverse data, and real-world constraints, the project can evolve into a deployable solution fit for modern audio enhancement demands.


